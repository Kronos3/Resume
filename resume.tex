% -- Encoding UTF-8 without BOM
% -- XeLaTeX => PDF (BIBER)

\documentclass[print]{cv-style}          % Add 'print' as an option into the square bracket to remove colours from this template for printing. 
                                    % Add 'espanol' as an option into the square bracket to change the date format of the Last Updated Text

\sethyphenation[variant=british]{english}{} % Add words between the {} to avoid them to be cut 
\usepackage{hyperref}% http://ctan.org/pkg/hyperref
\hypersetup{
  colorlinks=true,
  linkcolor=blue!50!red,
  urlcolor=green!70!black
}
\begin{document}

\header{Andrei}{ Tumbar}           % Your name

%----------------------------------------------------------------------------------------
%	SIDEBAR SECTION  -- In the aside, each new line forces a line break
%----------------------------------------------------------------------------------------

\begin{aside}
%
\section{Contact}
69 Congress Ave 
Rochester, NY
~
Mobile no:
607-379-0960
~
Email:-
andreitumbar@gmail.com
at1777@rit.edu
%
%
\section{Programming
   Languages}
C, Python, C++,
x86, Arm,
Cython,Verilog,
VHDL, Java, JavaScript,
CSS, HTML, Typescript
AJAX, \LaTeX{}
%
%
\section{Technologies}
Linux, cgroups,
AddressSanitizer,
FreeRTOS,
namespacing, socket,
Fuzzing, AFL,
SSL, libarchive,
CUDA, LLVM IR,
PostgreSQL, Django,
Google API,
Amazon AWS EC2,
CPython, libclang
%
\end{aside}


%----------------------------------------------------------------------------------------
%	WORK EXPERIENCE SECTION
%----------------------------------------------------------------------------------------

\section{Work Experience}

\begin{entrylist}
%------------------------------------------------
\entry
  {Aug-2020\\Aug-2021}
  {Jet Propulsion Laboratory - NASA}
  {Remote}
  {\jobtitle{Development Operations (Surface Ops)}\\
  Worked full time as a developer for the Mars-2020 project (Perseverance). Worked on SSim, a simulation program for modeling Mars-2020 rover flight software for rover planners to test their command sequences before instructing the real rover on Mars. Also worked on a new lunar arm mission planned for 2023 launch: COLDArm. Made the JPL's flight-software development framework, FPrime, usable as a simulation platform for deterministic development and mission operation planning.
  }
\entry
  {2018, 2019}
  {Cornell University}
  {Ithaca, New York}
  {\jobtitle{Programmer}\\
  Two summers working to implement statistical models in Mathematica and Python. Wrote front end to statistical modeling algorithms for research (letter of recommendation available upon request).}
  \entry
  {Jun-2019}
  {Treman State Park}
  {Ithaca, New York}
  {\jobtitle{Maintainance}\\
Maintained state park facilities. Work included cleaning bathrooms, planting trees, clearing trails, and general maintenance. Worked independently for the majority of the day while also in charge of one other worker.}
%------------------------------------------------

%------------------------------------------------
%------------------------------------------------

\end{entrylist}

%----------------------------------------------------------------------------------------
%	EDUCATION SECTION
%----------------------------------------------------------------------------------------

\section{Education}

\begin{entrylist}
%------------------------------------------------
{\vspace{-0.3cm}}
\entry
{2019--now}
{BS candidate {\normalfont in Computer Engineering (3.96/4.00 GPA) -- May 2023} \\
\begin{itemize}%[label=$\bullet$, leftmargin=*]
	\item Embedded and Realtime Systems
	\item Calculus I-IV, Uni-Phys I-II
	\item Linear Algebra \& Boundary Value+Diffeq \& Complex Variables
\end{itemize}
}
{Rochester Institute of Technology}
{\vspace{-0.3cm}}

%------------------------------------------------
\end{entrylist}

%----------------------------------------------------------------------------------------
%	OTHER QUALIFICATIONS SECTION
%----------------------------------------------------------------------------------------

{\vspace{-0.3cm}}
\section{Projects}

\begin{entrylist}
%------------------------------------------------
\entry
{2020-now}
{Phase Electron Microscope}
{Closed-source}
{Object-oriented approach to controlling various devices in data acquisition for a phase electron microscope. Designed so that any devices can be easily swapped out with very view changes to the overall system. Learned about programming hardware and performing image processing through GPU programming. Also learned how to program DAC on FPGA to control galvo mirrors controlling laser direction.}

\entry
{2014-now}
{AutoGentoo }
{github.com}
{A scalable Linux environment manager for creating optimized Gentoo Linux environments. Designed to bring higher performance to any platform at low maintenance cost.
\begin{itemize}
	\item FOSS implementation: \textbf{\href{https://github.com/AutoGentoo/AutoGentoo}{https://github.com/AutoGentoo/AutoGentoo}}
\end{itemize}
}

\entry
{2019-now}
{CPortage}
{github.com}
{A highly optimized rewrite of the Gentoo package manager. Written in C, CPortage is able to complete I/O, the most taxing phase of a package manager's calculations in less then a tenth of a second. Gentoo's custom language standards (ebuild) was rewritten in the Bison/Flex grammar parser.}

{\vspace{-0.3cm}}
%------------------------------------------------
\end{entrylist}

%----------------------------------------------------------------------------------------
%	INTERESTS SECTION
%----------------------------------------------------------------------------------------
\newpage
\section{General Interests}
  \vspace{-0.3cm}
\begin{entrylist}
%------------------------------------------------
\entry
{}
{Linux and Open-Source}
{}
{Love open-source and its development. Linux, being free, open-source and very customizable, has become a hobby of mine. }

\end{entrylist}\
\begin{entrylist}
%------------------------------------------------
\entry
{}
{C}
{}
{Advanced use of C and many of the POSIX technologies. Wrote networking, thread schedulers, grammar parsers, advanced data structures, tree recursion and more. Around 3+ years of experience.}
\end{entrylist}\
\begin{entrylist}
%------------------------------------------------
\entry
{}
{Python}
{}
{Python is a great tool for writing I/O scripts and complex string parsers. Also works well for front-end GUI design. Around 6+ years of experience.}

\end{entrylist}\
%----------------------------------------------------------------------------------------
%	SKILLS SECTION
%----------------------------------------------------------------------------------------

\section{Skills}
  \vspace{-0.2cm}

\begin{entrylist}
%------------------------------------------------
\entry
  {}
  {Time Management skills}
  {}
  {Able to keep up with full-time work at NASA as well as full-time student at RIT.}
%------------------------------------------------
\entry
  {}
  {Programming Ability }
  {}
  {Advanced programming skills in C and Python. Profficient in Java and C++.}
  
  \entry
  {}
  {Quick Learning Capability}
  {}
  {Able to learn new APIs and programming languages with ease.}
  
  \entry
  {}
  {Teamwork Skills}
  {}
  {Competed in Engineering and Computer Science competitions at the national level very successfully in multiple teams.}

  
%------------------------------------------------
%------------------------------------------------

\end{entrylist}

\end{document}
